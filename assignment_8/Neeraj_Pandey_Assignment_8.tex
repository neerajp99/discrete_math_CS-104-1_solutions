\documentclass{article}
\usepackage[utf8]{inputenc}
\usepackage{setspace}
\usepackage{amsmath}
\usepackage{amsthm}
\usepackage{amsfonts}
\usepackage{amssymb}
\usepackage{natbib}
\usepackage{graphicx}
\usepackage[dvipsnames]{xcolor}
\graphicspath{ {./images/} }
\usepackage[legalpaper, portrait, margin=0.9in]{geometry}
\title{ASSIGNMENT 8}
\author{Neeraj Pandey}
\date{Discrete Mathematics}


\begin{document}


\maketitle
\begin{flushleft}
\newcommand\tab[1][1cm]{\hspace*{#1}}
\textbf{Q1: } Prove that $\sqrt{p}$ is an irrational number for any prime $p$.
\newline
\newline
\textbf{Solution: } Let us consider $\sqrt{p}$ as rational.
\newline
Every rational number is of the form $\frac{a}{b}$
\newline
Squaring both the sides:
\[ p = a^2 + b^2\]
\[\implies a^2 = p \cdot b^2\]
\[\implies b|a^2\]
\newline
Using the fundamental theorem of arithmetic, there exists a prime number $k$ which divides $b$.
\newline
If $x|b \tab[0.1cm]{}\text{then} x|a^2 \tab[0.1cm] \text{and if}\tab[0.1cm] x|a^2 \tab[0.1cm]\text{then}\tab[0.2cm] x|a$.
\newline
\newline
$x|gcd(a, b) = 1$ is a contradiction unless b = 1 and a = 1.
\newline
\newline
Now, if b=1, there is no integer for which $a^2 = p$ and same for a = 1.
(where p is prime)
\newline
\newline
Therefore, $\sqrt{p}$ is irrational.
\newline
\newline

\end{flushleft}
\begin{flushleft}
\textbf{Q1: } Prove that $2^n \nmid n!$ for all $n \geq 1$.
\newline
\newline
\textbf{Solution: } According to Fundamental Theorem of Arithmetic:
\[n! = {2^{a_{1}}}\cdot{k_{2}^{k_{2}}}{.......}{k_{a}^{a_{a}}}\]
where $k_{1}, k_{2}....k_{k}$ are prime divisors,
\newline
$a_{1}$ should be $\geq n$:
\[ a_{1} = \small[\frac{n}{2}\small] + \small[\frac{n}{2^2}\small] + ....\]
\[ \small[\frac{n}{2}\small] + \small[\frac{n}{4}\small] + .....\]
\[< \frac{n}{2} \cdot \frac{1}{1-\frac{1}{2}}\]
\[\implies < n\]
\newline
\newline
Hence proved.
\end{flushleft}





\end{document}
