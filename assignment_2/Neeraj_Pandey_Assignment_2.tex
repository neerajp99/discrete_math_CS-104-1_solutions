\documentclass{article}
\usepackage[utf8]{inputenc}
\usepackage{setspace}
\usepackage{amsmath}
\usepackage{amsthm}
\usepackage{amsfonts}
\usepackage{amssymb}
\usepackage{natbib}
\usepackage{graphicx}
\usepackage[legalpaper, portrait, margin=0.9in]{geometry}


\title{ASSIGNMENT 2}
\author{Neeraj Pandey }
\date{Discrete Mathematics}



\begin{document}

\maketitle
\textbf{Q1.} Show that among $n + 1$ positive integers of which is greater than $2n$, there are two that are relatively prime. (Two numbers are said to be relatively prime if their greatest common divisor is 1).
\newline
\textbf{Solution: } Let us proof by using a Lemma,
\newline
\textbf{Lemma: } Every consecutive positive integer is prime. Proof by contradiction says that n and n+1 are positive integers such that prime number divides both of them.
\newline
\textbf{Contradiction:} k | 1, but k $>$ 1
By lemma stated above, divide the $2n$ numbers into the following sets,
\newline
\[ \small\{ 1, 2\small\}\]\\
\[ \small\{ 3, 4\small\}\]\\
\[ \small\{ 5, 6\small\}\]\\
\[ \small\{ ...\small\}\]\\
\[ \small\{ ... \small\}\]\\
\[ \small\{ 2n - 1, 2n\small\}\]
\newline
As there are n such sets and every number $\leq$ 2n, there are $n+1$ numbers and $n$ sets. By Pigeonhole Principle there is atleast one set which has atleat 2 elements. As elements of this set are consecutive positive integers, therefore by Lemma they are relatively prime.
\newpage
\begin{flushleft}
\textbf{Q2.}  Suppose that we have n natural numbers none of which is greater than $2n$ such that the least common multiple of any two is greater than $2n$. Show that all n numbers are greater than \(\frac{2n}{3}\).
\end{flushleft}
\begin{flushleft}
\newcommand\tab[1][1cm]{\hspace*{#1}}
\textbf{Solution:} Let us proof this by contradiction:
\newline
\newline
 Let's say that there is a number $a_{1}$ among all the chosen numbers \(\ni\) $ a_{1}$\textless \( 2_{n}\) and the rest of the numbers as $a_{i}$\textgreater \(\frac{2n}{3}\).

We can see that following property from the above:
\[ 2a_{1} < 3a_{1} < 2_{n}\]\\
Take a set, S = \small\{ $2a_{1}, 3a_{1}, a_{2} .... a_{n}$ \small\}
\newline
here, we will take a helping theorem (lemma) \\
\textbf{Lemma 1: }There are 2 positive integers among n + 1 which divides the other and is greater than $2n$.\\
So, we can have the following sets:
\[ \small\{ 1, 2, 4, 8 ...\small\}\]
\[....\]
\[....\]
\[\small\{2n-1, 4n-2, 8n-4 ...\small\}\]
$\implies$ These are the following properties of the above sets:\\
\textbf{1.} There are 'n' sets in total.\\
\textbf{2. }There are 2 numbers from a set which divides one from the other.\\
\textbf{3. }Numbers smaller than 2n fall under these sets.
\newline
- From the lemma used above, there are 2 numbers that one divides from the other. As, $2a_{1}$
/ $3a_{1}$ is not an integer, we can conclude that there are 2 numbers $a_{i}$ and $a_{k}$ \(\ni\) $a_{k} > a_{i}$ and $a_{i}$ divides $a_{k}$.
\newline
Therfore, if $a_{1} < 2n$, there there exist 2 numbers among \small\{ $a_{1}, a_{2}....a_{n}$ \small\} and their LCM is smaller than 3.
Hence, proved by Contradiction.


\end{flushleft}


\end{document}
