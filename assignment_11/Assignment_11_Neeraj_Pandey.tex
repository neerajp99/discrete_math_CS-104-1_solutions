\documentclass{article}
\usepackage[utf8]{inputenc}
\usepackage{setspace}
\usepackage{amsmath}
\usepackage{amsthm}
\usepackage{amsfonts}
\usepackage{amssymb}
\usepackage{natbib}
\usepackage{graphicx}
\usepackage[dvipsnames]{xcolor}
\graphicspath{ {./images/} }
\usepackage[legalpaper, portrait, margin=0.9in]{geometry}
\title{ASSIGNMENT 11}
\author{Neeraj Pandey}
\date{Discrete Mathematics}


\begin{document}


\maketitle
\begin{flushleft}
\newcommand\tab[1][1cm]{\hspace*{#1}}
\textbf{Q1(a): } There are 10 telegrams and 2 messenger boys. In how many different ways can the telegrams be
distributed to the messenger boys if the telegrams are distinguishable?
\newline
\newline
\textbf{Solution: } Every telegram has has 2 messenger boys to choose from the given 10 telegrams. We need to find the number of subsets:
\[ S = S_{1} x S_{2} .... x S_{10}\]
\[\implies 2^{10}\]
\newline
So, there are $2^{10}$ different ways in which 10 telegrams can be distributed among the messenger boys.
\newline
\newline
\textbf{Q1(b): } In how many different ways can the telegrams be distributed to the messenger boys and then
delivered to 10 different people if the telegrams are distinguishable?
\newline
\newline

\textbf{Solution: }$10$ telegrams of $n$ different ways can be distributed among 10 different boys in $n * 10!$ ways.
As there are 2 boys among which these telegrams have to be distributed, so:
\[2^{10} * 10!\]
So, there are $2^{10} * 10!$ ways in which these telegrams can be distributed if the telegrams are distinguishable.
\newline 
\newline
\textbf{Q1(c): } Solve (a) under the assumption that telegrams are indistinguishable.
\newline
\newline 
A messenger can get either $0, 1, 2, 3.....,10$ telegrams if all telegrams are indistinguishable, which adds up to 11 ways.
\newline
\newline

\textbf{Q2(a): } Find the sum of all 4-digit numbers that can be obtained by using the digits 2, 3, 5 and 7?
\newline
\newline
\textbf{Solution: } As the number is a 4-digit, so we have 4 numbers. So, there are a total of $256$ different ways in which these 4 numbers can be represented.\\
Also, these number can be in units, tens, hundreds or thousands:
\[\implies 256/4 = 64\]
\[= 17 * 6(64 + 17) * 64 * (10 + 17) * 64 * 1000\]
\[= 1208768\]
Therefore, the sum of all 4-digit numbers that can be obtained by using the digits 2, 3, 5 and 7 is 1208768.

\textbf{Q2(b): } Find the sum of all 4-digit numbers that can be obtained by using the digits 2, 3, 5 and 7 and no digit is repeated?
\newline
\newline
\textbf{Solution: } There are 4 numbers and no digit is repeated so, we have:
\[4 * 3 * 2 * 1\]
\[= 24 \text{ ways}\]
Also, these number can be in units, tens, hundreds or thousands:
\[\implies 3 * 2 * 1\]
\[ = 6\]
So, there are:
\[17 * (6 + 17) * 6 * (10 + 17) * 6 * (100 + 17) * 6 * 1000\]
\[= 113322\]

Therefore, the sum of all 4-digit numbers that can be obtained by using the digits 2, 3, 5 and 7 and no digit
is repeated is 113322.

\end{flushleft} 
\begin{flushleft}

\end{flushleft}





\end{document}